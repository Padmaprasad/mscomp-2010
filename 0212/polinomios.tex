\documentclass[12pt]{article}
\usepackage{amsmath,amssymb}

\begin{document}
\noindent{}\hrulefill{}
\begin{flushright}
  \textbf{\large{Problem set \#5}}
\end{flushright}
\tableofcontents{}
\noindent{}\hrulefill{}
\section{Polinomios}
En los test de $\chi^2$ suelen surgir ecuaciones del tipo
\begin{align}
\label{eq:1}
  x^2\sum_{i=1}^n\frac{1}{x+r_i}=k\,.
\end{align}
Dicha ecuaci\'on se puede escribir como una ecuaci\'on polin\'omica para $x>0$ y $r_i>0$. de manera que pueda utilizarse la clase de polinomios de Numpy, \texttt{np.poly1d}. La ec.~(\ref{eq:1}), puede reescribirse como la ecuaci\'on polin\'omica equivalente:
\begin{align}
  x^2P(x)=k Q(x)
\end{align}
donde
\begin{align}
  Q(x)=\prod_{i=1}^n(x+r_i)\,,
\end{align}
y
\begin{align}
  P(x)=\sum_{i=1}^n\prod_{j\neq i}(x+r_j)\,;
\end{align}
para finalmente construir la ecuaci\'on polin\'omica
\begin{align}
  R(x)= x^2P(x)-k Q(x)=0\,.
\end{align}
Tenga en cuenta que Numpy puede sumar, multiplicar, integrar, etc polinomios, pero no puede dividirlos. Note que los polinomios en Numpy se pueden generar a partir de los coeficientes o las ra\'\i ces. Y finalmente, que  los polinomios de Numpy, \texttt{P}, tienen un atributo \texttt{P.r} que entrega las ra\'\i ces.  

\end{document}
